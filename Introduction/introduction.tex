%%% Thesis Introduction --------------------------------------------------
\chapter{Introduction}
\ifpdf
    \graphicspath{{Introduction/IntroductionFigs/PNG/}{Introduction/IntroductionFigs/PDF/}{Introduction/IntroductionFigs/}}
\else
    \graphicspath{{Introduction/IntroductionFigs/EPS/}{Introduction/IntroductionFigs/}}
\fi

\section{Motivation}
Modern software systems are hugely complex. This causes a lack of understanding of the underlying components which in turn poses problems in the design, development and maintenance of such systems. The modern methods of dealing with this complexity in the software engineering discipline include the defining of short iterative `agile' development processes, automated unit testing, continous integration, application of design patterns and principles, and continuous refactoring. In addition to this, there is a requirement to measure the system using clearly defined rules and analyses to establish a picture of overall software quality. Visualisation is a technique for gaining insight into these measurements through graphical representation and interaction. There is a need for more kinds of visualisations in order to interpret the complex data produced from software and its measurements.

Tag clouds are a common example of information visualisation frequently used on the web. They are used in social sites to highlight important, prominent keywords and show the relative importance of tags. There have been numerous studies evaluating tag cloud usage as an visualisation technique, and proposing modifications to algorithms and layout \citep[e.g.][]{rivadeneira07, rivadeneira07, Oosterman10, lee10}.  Previous evaluations comparing the effectiveness of tag clouds to lists or tables suggest tag cloud performance is inferior for specific user tasks \cite{Oosterman10}. However, there has not been any research evaluating the performance of tag clouds for heterogeneous and multivariate data, such as found in software engineering. The complex nature of this data is not easily displayed in a simple visualisation such as a table.

\section{Thesis Statement}

There are many obstacles to producing an effective visualisation of software elements and measurements; these problems are related to scale, size and complexity. The fast rate of change and dynamics of teamwork also add challenges. Many different approaches which have been customised to a specific problem or task have been produced. 

This research will draw on previous tag cloud studies and apply that knowledge to the software engineering domain, specifically exploring whether a tag cloud visualisation of software measurements can assist software engineering user tasks. This can be divided into areas which warrant further consideration:

\begin{itemize}
	\item Knowledge of the types of information that are likely to be useful to developers
	\item Identification of user tasks that represent meaningful ways users may interact with the data
	\item Being able to measure how beneficial the visualisation is for comparison against a benchmark
	\item Evaluation of the visualisation in a realistic scenario
\end{itemize}

The primary research question we are pursuing is to discover if tag cloud visualisations may usefully be applied to software.  Previous evaluations comparing the effectiveness of tag clouds to a baseline visualisation of tables suggested tag cloud performance was inferior for user tasks of searching and attribute matching \cite{Oosterman10}. However, there has not been any research evaluating the performance of tag clouds for heterogeneous and multivariate data, such as found in software engineering. The complex nature of this data is not easily displayed in a simple visualisation such as a table.

We hypothesise that visualizing relevant software data elements, their properties and their relationships via a tag cloud visualisation can promote a greater understanding of a software system. We also expect the visualisation will assist users in completing specific types of tasks, appropriate to the domain of software engineering.

\section{Objectives}

Within the primary research question there are more specific areas of interest whose potential can be explored:

\begin{itemize}
	\item Tag clouds as a visualisation technique for typical user tasks such as software data navigation, exploration and searching. Exploring tag clouds used as a table of contents or index into other visualisations.
	\item User interaction and customisation. This could include for example removing irrelevant data elements, tailoring the tag length or respecting user selected tag positioning.
	\item Visualisation of software metrics within tag clouds. Assessing the presence or absence of object-oriented design patterns or code smells in data. Presentation of conflicts in design principles.
	\item Showing relationships between relevant data elements.
	\item Visualisation of the evolution of software data structures over time.
	\item Developing guidelines for visualisation designers. This could include defining when tag clouds may be a better choice than other visualisation techniques, what tag cloud layout is most appropriate for a given task and the data set size which yields optimal results.
\end{itemize}

\section{Contributions}

\section{Outline}
This thesis presents the following relevant points:

\begin{itemize}
	\item Describing current software engineering practices, how they attempt to address quality issues and manage the manifold complexities existing in today's software systems.
	\item Discussion of software quality measurement through collection of metric data and static or dynamic analysis. 
	\item Presenting visualisation as a technique for interpreting the multivariate data found in software engineering.
	\item Introduction of tag cloud visualisation. Discussion of related research both suggesting modifications for improvement, and evaluating tag cloud usage for generic user tasks. 
	\item Limitations of current research of tag clouds visualising software properties. Discussion of issues with existing tag cloud implementations in the software engineering domain and presenting further points to consider when developing a tag cloud visualisation prototype.
	\item Examples of approaches that could be taken for evaluating a prototype.	
	\item Primary research question and hypothesis. Specific areas of interest to be explored and presentation of the research approach.
\end{itemize}

%%% ----------------------------------------------------------------------


%%% Local Variables: 
%%% mode: latex
%%% TeX-master: "../thesis"
%%% End: 
