
Outlining techniques used to visualise structural and behavioural software information. How do these techniques compare with the classifications used to describe information visualisation and the techniques used in multi-variate data visualisation? What kinds of tasks does each type of visualisation best support?

\begin{itemize}
	\item Graph layouts – Where to place data nodes for maximum readability and effectiveness. E.g. tree layout (cone tree), hierarchical layout of directed graphs (call graph), orthogonal layout of planar and general graphs (UML class diagram), force-directed layout (spring embedder), hyperbolic, cluster graphs etc.
	\item Space filling technique – presenting maximum amount of structured information in minimum amount of space e.g. Tree-map, sunburst.
	\item Information murals – large information space fitting entirely within a view. Useful for visualising trends and patterns in the overall distribution of information. E.g. line and pixel representation, execution mural.
	\item Interaction and navigation – due to the dense structure of the data (huge size and possible high dimensionality) there are difficulties in presenting data in a single view. Hence techniques such as scrolling and panning, focus and context views (fish-eye, magic lens), multiple views.
	\item Animation – displaying transitions between different states of the program.
	\item Virtual environments
	\item Auralisation
\end{itemize}

Software visualisation systems - tools, toolkits and libraries that generate visualisations of software and software artefacts.

% ------------------------------------------------------------------------

%%% Local Variables: 
%%% mode: latex
%%% TeX-master: "../thesis"
%%% End: 
