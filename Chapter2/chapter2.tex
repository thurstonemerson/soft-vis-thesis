\chapter{Considerations in Tag Cloud Visualisation}
\ifpdf
    \graphicspath{{Chapter2/Chapter2Figs/PNG/}{Chapter2/Chapter2Figs/PDF/}{Chapter2/Chapter2Figs/}}
\else
    \graphicspath{{Chapter2/Chapter2Figs/EPS/}{Chapter2/Chapter2Figs/}}
\fi

In information visualisation research, there have been various attempts to extend and modify the tag cloud metaphor for novel purposes \citep[such as][]{Candan08, Cui10, dicaro08, lee10} . However, it is not clear what gains application of tag clouds can reasonably achieve without a thorough understanding of the factors which influence user perception and a clearly defined knowledge of the kinds of tasks where tag clouds are shown to be high-performing. This chapter seeks to set out the current knowledge surrounding the usage of tag clouds, identifying areas which are either unknown or where weak evidence has been provided. These areas are then clarified with eye-tracking experiments.

\section{Tag clouds in information visualisation}

\begin{itemize}
	\item Discuss where tag clouds sit in the field of information visualisation. Describe the classification of information visualisation types e.g. one/two dimensional, multi-dimensional, text-web, graphs. Visualisation technique (Keim 2002) e.g. 2D/3D, geometrically-transformed, iconic display, dense pixel, stacked display etc. 

	\item Note that ordinary tag clouds provide limited interaction, although some novel approaches have attempted to rectify this.
\end{itemize}


% ------------------------------------------------------------------------

%%% Local Variables: 
%%% mode: latex
%%% TeX-master: "../thesis"
%%% End: 


\section{Taxonomy of tag cloud features}

Presenting a taxonomy of tag cloud visual features which can reasonably be mapped to dimensions in a dataset.

Font and text properties - taken from CSS font and text properties W3C defined.

\subsection{Font family}
Specific names or generic (e.g. serif, sans-serif, cursive etc.)

\subsection{Font size}

\subsection{Font style}
Normal, italic, oblique.

\subsection{Font weight}
Normal vs bold.

\subsection{Colour distribution}

\subsection{Intensity contrast or saturation}

\subsection{Colour}

\subsection{Decoration}
Underline, over-line, line-through.

\subsection{Other properties which can''t reasonably be mapped}
Font-variant - allowing small caps. Letter spacing. Line height. All capitals. White space. Word spacing.


% ------------------------------------------------------------------------

%%% Local Variables: 
%%% mode: latex
%%% TeX-master: "../thesis"
%%% End: 


\section{Factors influencing visual perception}

Additional factors with influence user visual perception.

\begin{itemize}
	\item Tag length
	\item Number of characters
	\item White space
	\item Position with tag cloud (e.g. quadrant)
	\item Tag area
	\item Proximity effect
	\item Semantic clustering
	\item Layout - Spiral, horizontal, alphabetical sorting, random, vertical,combination, clustered (statistically, semantically), typewriter, shape-filled.
\end{itemize}

\subsection{Order of importance}

Noting the order of importance for visual features according to both tag cloud literature and any other known relevant physiological and psychological information. 

\subsection{Inter-dependencies of properties}

\subsection{Literature review}

Review literature discussing these factors and identify gaps in the knowledge or places where there exists a lack of hard evidence.


% ------------------------------------------------------------------------

%%% Local Variables: 
%%% mode: latex
%%% TeX-master: "../thesis"
%%% End: 


\section{Usability of the tag cloud representation}

Setting out a (generally accepted) criteria for evaluating usability in information visualisation. (Freitas et. al).

\begin{itemize}
	\item Completeness: representing all the semantic contents of the data to be displayed.
	\item Spatial organisation: Overall layout. How easy it is to retrieve information and be aware of overall distribution of data.
	\item Information coding: additional characteristics used to aid user perception.
	\item State transition: rebuilding visual representation after a user action. 
\end{itemize}

Reviewing literature discussing this usability criteria in tag clouds and identifying gaps in the knowledge.

% ------------------------------------------------------------------------

%%% Local Variables: 
%%% mode: latex
%%% TeX-master: "../thesis"
%%% End: 


\section{Task-oriented analysis}

Setting out a (generally accepted) set of tasks that tag clouds support. (Rivadeneira et al).

\begin{itemize}
	\item Searching: Locating a specific data element.
	\item Browsing: Information searching with no specific target item in mind.
	\item Overview/Gisting: Forming a general impression of the underlying dataset.
	\item Recognition/matching: recognising which of several sets of information a tag cloud is likely to represent.
\end{itemize}

Reviewing literature evaluating tag cloud performance in the area of tag clouds from a task perspective and identifying gaps in the knowledge or where there exists a lack of hard evidence.

\subsection{Searching}

Research supporting premise that tag clouds provide sub-optimal support for search tasks
- Oosterman and Cockburn (2010), tables faster and more accurate than tag clouds
- Halvey and Keane (2007), lists perform better than tag clouds
- Kuo et al (2007), lists more helpful and faster than tag clouds
- Sinclair and Cardew-Hall (2008), users prefer traditional search interface

Research supporting premise that tag clouds provide support for search tasks
- Kuo et al (2007), reported user satisfaction higher with tag clouds than with lists

\subsection{Browsing}

Research supporting premise that tag clouds provide support for browsing tasks
- Sinclair and Cardew-Hall (2008), users prefer a tag cloud over a traditional search interface

\subsection{Gisting}

Research supporting premise that tag clouds provide sub-optimal support for gisting
-  Rivadeneira et al (2007), lists may provide a more accurate impression than tag clouds

Research supporting premise that tag clouds provide support for gisting
- Kuo et al (2007), tag clouds provide improvements for summarising descriptive information

\subsection{Information Recognition/Matching}

Nothing found yet

% ------------------------------------------------------------------------

%%% Local Variables: 
%%% mode: latex
%%% TeX-master: "../thesis"
%%% End: 


\section{Evaluation}

Possibility of controlled experiments using eye-tracking data in those (relevant) areas where knowledge gaps have been identified or hard evidence is lacking. 

\begin{itemize}
	\item The influence of tag cloud visual features on perception.
	\item Usability of tag clouds as an information visualisation technique.
	\item Assessing the support of tag clouds for various task types.
\end{itemize}


% ------------------------------------------------------------------------

%%% Local Variables: 
%%% mode: latex
%%% TeX-master: "../thesis"
%%% End: 



% ------------------------------------------------------------------------


%%% Local Variables: 
%%% mode: latex
%%% TeX-master: "../thesis"
%%% End: 
