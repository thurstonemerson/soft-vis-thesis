\chapter{Appendix A}

\section{Tools for Obtaining Software Metric Data}

\subsection{Eclipse Metrics Plugin}

\begin{itemize}

	\item \textit{Metrics}\footnote{\url{http://metrics.sourceforge.net/}}: Discontinued. Contains CK metric suite and others.

	\item\textit{Metrics2}\footnote{\url{http://metrics2.sourceforge.net/}}:  Continuation of above metrics project.

	\item \textit{Google CodePro AnalytiX}\footnote{\url{http://code.google.com/javadevtools/codepro/doc/index.html}}: Contains a variety of metrics but does not appear to contain CK metric suite. Contains Halstead Science metrics.

	\item \textit{GERT(Good Enough Reliability Tool)} (2006): Developed at the North Carolina State University, this plugin is using the STREW metric suite, containing a number of complexity, OO and size adjustment metrics.

	\item \textit{Checkstyle}

\end{itemize}

\subsection{Software Analysis Platform} 

\begin{itemize}

	\item \textit{Alitheia Core}\footnote{\url{http://istlab.dmst.aueb.gr/~george/pubs/2009-ICSERD-GS/poster.pdf}}: Developed at Athens University, this is a platform for software quality analysis designed for research on large data sources. Works by importing source, mailing lists, bugs etc into a database and doing data preprocessing and metadata extraction. Researchers create analysis plugins by implementing an interface. 

	\item \textit{MASU platform}\footnote{\url{http://masu.sourceforge.net/}}: Developed at Osaka University, this analysis platform calculates CK Metric suite and cyclomatic complexity metrics for a variety of different programming languages.

	\item \textit{Sonar platform} \footnote{\url{http://www.sonarsource.org/}}\footnote{\url{http://docs.codehaus.org/display/SONAR/Documentation}}: Sonar is an open-source web based code quality analysis tool for Maven-based Java projects. It covers a wide area of code quality check points and is possible to extend via a plugin mechanism. Sonar already contains coverage clouds using class names in a limited capacity. It is possible to collect metrics generated by a 3rd party source and inject into Sonar to visualise. It is also possible to compile a custom metric.

	\item \textit{Moose}\footnote{\url{http://www.moosetechnology.org/}}

	\item \textit{Borland Together} 

\end{itemize}

\subsection{Metric frameworks/tools}

Command line tools or libraries that can generate metrics.

\begin{itemize}

	\item \textit{CKJM}\footnote{\url{http://www.spinellis.gr/sw/ckjm/}} (2005):  An open source java tool for calculating CK metrics suite, CA and NPM. This is a simple command line tool, and is also an ant task and maven plugin. 

	\item \textit{CKJM Pro}\footnote{\url{http://gromit.iiar.pwr.wroc.pl/p_inf/ckjm/}} (2010): An extended version of CKJM which generates additional metrics.

\end{itemize}

\section{Metric Output from Software Analysis Tools}

\begin{itemize}
	\item Sonar- PDF, HTML or CSV
	\item CYVIS - CSV, XML
	\item Rational software analyser - XML, PDF or HTML
	\item Moose - CSV, XML
	\item Eclipse metrics2 - XML
	\item RSM - HTML, CSV, XML
	\item Borland Together - XML, HTML, CSV, Tab separated
	\item SourceMonitor - XML, CSV
	\item State of Flow - HTML, CSV or XML
\end{itemize}


% ------------------------------------------------------------------------

%%% Local Variables: 
%%% mode: latex
%%% TeX-master: "../thesis"
%%% End: 
