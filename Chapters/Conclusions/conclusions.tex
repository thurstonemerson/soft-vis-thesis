\def\baselinestretch{1}
\chapter{Conclusions} \label{chap:conclusions}
\ifpdf
    \graphicspath{{Chapters/Conclusions/ConclusionsFigs/PNG/}{Chapters/Conclusions/ConclusionsFigs/PDF/}{Chapters/Conclusions/ConclusionsFigs/}}
\else
    \graphicspath{{Chapters/Conclusions/ConclusionsFigs/EPS/}{Chapters/Conclusions/ConclusionsFigs/}}
\fi

\def\baselinestretch{1.66}

There is a need in software engineering to explore new visualisation techniques with a low-level of conceptual complexity, and for the effectiveness of visualisation tools to be adequately quantified. This thesis investigated the application of the tag cloud metaphor to the software engineering domain and identified limitations in current research evaluating tag cloud visualisation; a lack of research evaluating visualisation tools and techniques for domains outside of the web and user-generated data domains, and a limited range of evaluation approaches. In particular, there is a dearth of research utilising evaluation strategies which focus on visual data analysis and reasoning, or collaborative data analysis. These strategies have a goal of understanding the underlying data exploration process of a tool (as opposed to usability of a system slice or technique), and are of high relevance and practical value in the information visualisation field. This is also of special importance in software engineering, as demonstrations of visualisation benefits exploring data using realistic scenarios may be a key factor contributing to the lack of visualisation techniques being integrated into mainstream development environments. In response, this thesis contributes a new system for interacting with software engineering or other multi-variate data using tag clouds, and evaluates both utilised enhanced tag cloud features, and the knowledge discovery process within the system itself. 

\section{Review of thesis contributions}

The central theme in this thesis is the design and evaluation of a visualisation system that utilises tag clouds, a highly recognisable visualisation with a low-level of conceptual complexity, and that supports exploration of multi-variate data such as that found in software engineering. To that end, we analysed the challenges in visualising multi-variate data, and the capabilities of tag clouds (Chapters~\ref{chap:tagcloud}) --- in particular necessary task types and the effects on user perception for available visual properties ---  to develop design considerations guiding the development of an interactive tag cloud visualisation system. 

We applied these considerations to the design of Taggle (Chapter~\ref{chap:taggle}), a Java-based tag cloud visualisation system utilising enhanced tag cloud features to explore multi-variate software measurements. We conducted a heuristic evaluation of Taggle using domain experts (Chapter~\ref{chap:heuristiceval}). This revealed that the tag cloud technique of contrasting visual font properties mapped to data fields was felt to be instinctively comprehensible, and underlying data distributions, outliers and tags clusters were able to be inferred. However, it was considered the amount of information that could be interpreted was greatly dependent on the software's support for selection of appropriate mappings. Following the heuristic evaluation and before the experiments, alterations were made to the software to improve the system's usability and appropriateness of exploration of multi-variate data. Thus the experiment subjects gained the benefit of the suggested improvements.

We performed a systematic mapping study synthesising existing tag cloud evaluation research (Chapter~\ref{chap:strateval}). This served to provide an overview of existing evaluations of the tag cloud visualisation and tools which incorporated the technique, and mapped evaluation approaches and methods, as well as the target domain. We discovered most research focused on interactive interfaces for special datasets, mediums or populations but utilised a limited range of evaluation approaches -- focusing on \emph{visualisation use} strategies which measured timed user performance or subjective experience, as opposed to insight or knowledge gathering support. No interface identified in the mapping study proposed a system such as Taggle, where data fields from a multi-variate dataset are mapped to tag cloud visual properties and manipulated interactively. Tag cloud visualisation itself had not been as extensively evaluated as other areas, indicating there was still room to define their overall effectiveness and develop ways to improve the tag cloud as a technique. Furthermore, research was significantly skewed towards web and user generated data domains, with only one paper evaluating a tag cloud visualisation system in the software engineering field. 

Based on the systematic mapping study results, a series of targeted evaluations was planned and conducted to explore the potentials and limitations of Taggle  (Chapter~\ref{chap:eval}). In order to obtain a broad-based investigation of points of relevance for both the tag cloud technique and our interactive tool, the experiments were conducted in both areas of \emph{visualisation use} (the tag cloud technique) and \emph{data process analysis} (a whole-tool approach focused on the knowledge discovery process). Two \emph{user performance} experiments were conducted where we examined the enhanced tag cloud features included in Taggle to see if they improved user performance in visual search tasks (Chapters~\ref{chap:exp1} and \ref{chap:exp2}). One \emph{visual data analysis and reasoning} evaluation was conducted where we examined data exploration and knowledge discovery support in Taggle (Chapter~\ref{chap:exp3}).


%the background colour expt is helpful for DOI based vis where small tags are inevitable; similarly dual mappings are good for correlation spotting particularly when involving small tags

\paragraph{Tag background colour} We compared user visual search response times alternating foreground and background colour in target tags. Results indicated usage of tag background colour as a data variable field can produce faster search response times than foreground colour when the target tag is small. Small tags are inevitable in tag cloud visualisation when using a full range of tag sizes, and when visualising large datasets, such as those found in software engineering. We think providing the background colour option as the default setting in Taggle may help users identify mapped variables (particularly for small or iconified tags), and explore correlations more efficiently between data variables represented by size and colour. 

\paragraph{Dual data mappings} We compared user visual search response times alternating single mappings (colour or font size) and dual mappings (colour and font size together) in target tags. Results indicated dual mappings of font size and colour to a data variable field can produce faster visual search response times than font size or colour alone. We think provision of multiple data mapping options to tag cloud visual properties in the Taggle interface highlights and reinforces the data mappings: this may help users identify data correlations and explore the data more effectively.

\paragraph{Visual search patterns within a tag cloud} Eye-gaze data was collected from experiments where participants were executing a visual search within a tag cloud. Our eye-tracking data analysis showed that the introduction of a visual property hint when performing a search task can alter the search strategy to the eye scan path focusing on tags with the target mapping (efficient feature search). When elements such as target category size or tag length increased the complexity of the task, users generally employed combination visual search methods in a tag cloud, switching between visual feature search, serial scanning and chaotic search methods.

\paragraph{Knowledge discovery} We evaluated Taggle's support for data exploration and knowledge discovery through an empirical user study, where participants explored an unknown software engineering dataset and attempted to complete a set of benchmark tasks. Results indicated Taggle could be successfully used by people with minimal training to discover relevant information in a dataset. Experiment participants were able to effectively complete visual classification, clustering, association and summarising tasks. Eye-gaze and mouse-tracking data showed participants utilising the data summary panel and features such as mapping multiple visual properties to data fields, and static and dynamic filtering to analyse and interpret data.  

\section{Limitations and future work}

We hope this thesis will serve as a prelude to a continuing stream of research investigating the limits and potentials of tag clouds in both general information and software engineering visualisation. In this section we elaborate some of the limitations of our experiments and opportunities for future research. 

\begin{description}
	\item [Spiral versus typewriter layout] Both \emph{visualisation use} experiments on a static tag cloud produced irregular results between the spiral and typewriter layouts and we were unable to determine the reasons behind this. It is possible that there are inherent differences in the way users search for visual targets in a spiral layout, or it may be something particular to the spiral layout algorithm used in Taggle. Future eye-tracking data analysis or experiments may focus on discovering more about this difference.

	\item [Visual search patterns and task complexity] When elements such as target category size or tag length increased the complexity of the task, users generally employed combination visual search methods in a tag cloud. For example, varying target category size over experiment repetitions produced some abnormally long visual search response times for categories with greater sizes. It would be interesting to focus on category size (or task complexity) as an experiment factor in exploring eye-scanning methods and to discover at what point (such as overall percentage or specific number of tags) participants start producing atypical response times. Response times for varying category or dataset sizes are particularly pertinent in the software engineering domain where dataset size can be a challenging element.

	\item [Temporal data] A limitation of our knowledge discovery experimentation with Taggle is that temporal data was absent and therefore trend analysis was not included with the other benchmark task types. There are a variety of ways in which temporal data can be analysed using tag cloud visualisation (for example comparing multiple clouds or using visual properties such as order to display time, as shown in Figure~\vref{fig:tagcorrelation}, or evolving clouds that dynamically change in real time). The effectiveness of tag cloud visualisation for trend analysis is yet to be determined.

	\item [Use of interface features to assist mapping choices] Exploratory analysis of eye-gaze data for the knowledge discovery experiment showed the summary data screen was used extensively. Other information displaying parts of the interface such as the colour chip legend and status bar were used minimally or not at all. This could drive another round of interface refinement such as the one that happened after the heuristic evaluation. Further analysis of the eye-gaze data may be done for other interface features as well as to look carefully at the process of how the summary data screen was used for tasks. 

	\item [Taggle comparative evaluation] We can use the same data, taskset and NASA-TLX questionnaire to gather results from another information visualisation tool. These results can be used in order to compare the task completion and perceived workload of Taggle to other software engineering visualisation tools.

\end{description}

\section{Closing remarks}

Visualisation allows us to gain insight into the screeds of information which are available to us in ever-increasing amounts. Modern software has an inherent scale and complexity, constantly evolving, with complex relationships between components. Despite the plethora of techniques proposed for visualising software, they have not been widely applied or integrated into mainstream development environments. In software engineering, there is a need to explore new visualisation techniques --- particularly those such as tag clouds which have an apparent low-level of conceptual complexity, and furthermore, for the effectiveness of such techniques and tools to be demonstrated. This thesis has presented \emph{Taggle}, an interactive visualisation tool utilising tag clouds. Through careful consideration of applicable perceptual factors, Taggle is particularly suited to software engineering data. Empirical evidence suggests the system is useful for visual classification, clustering, association and summarising tasks. This thesis illustrates the greater potentials of the common tag cloud in exploring and making sense of multi-variate data. 


% ------------------------------------------------------------------------

%%% Local Variables: 
%%% mode: latex
%%% TeX-master: "../thesis"
%%% End: 
