\chapter{Empirical Feasibility Study}
\ifpdf
    \graphicspath{{Chapter6/Chapter6Figs/PNG/}{Chapter6/Chapter6Figs/PDF/}{Chapter6/Chapter6Figs/}}
\else
    \graphicspath{{Chapter6/Chapter5Figs/EPS/}{Chapter6/Chapter6Figs/}}
\fi

Where are tag clouds best or other visualisations?

\section{Apparatus and Materials}

Tobii eye tracker (version) along with the Tobii Studio 1.x software was used to record and replay participants' eye movements. Using Tobii Studio 1.x, gaze plots were created and a video recording of the screen with eye movement data was played back. Image vs Live Application Issues. Resolution and Analysis.

\section{Participants}

\begin{itemize}
	\item Size - between 5-15 participants for each condition to be tested (compared).
	\item Recruiting participants using convenience sampling.  Spanned locations for data collection to provide access to a wide variety of potential participants. Offering of small incentive.  Conditions to be used when recruiting – e.g. must be familiar with software engineering.
	\item Ensuring homogeneity between participants – pre-test questionnaire. E.g. are you familiar with tag clouds? One way analysis of variances (ANOVA’s) looking for statistical differences between groups in relevant experiences. Ensure that similar levels of experience are found within the given factors. Ensure reasonably even distribution of sexes.
	\item Creation of randomised list of conditions to be compared created, each person tested in the random choice of condition next on the list.  Equal numbers of participants for each condition.
\end{itemize}

\section{Procedure}

Researcher uses interview script (created in such a way as to ensure consistency between test conditions) as guidance.

\begin{enumerate}
	\item Screening questions asked to ensure suitability for participation.
	\item Sign audio and video release form since comments and eye movements are recorded.
	\item Study introduction and brief explanation of test.
	\item Calibration of eye-tracker.
	\item Participants provided with tasks and begin completing them.
	\item Upon completion of the tasks, participants are a video playback of the task they had just completed along with their eye movements superimposed on the screen.  At the same time they are asked to talk about the task they had just completed. (Users can fast-forward, rewind or pause the video and can use mouse to point at items on the screen. Video shown at half speed).  
\end{enumerate}

\section{User Tasks}

\subsection{Analysis}
Aligned to impression forming task from \cite{rivadeneira07}.

\subsection{Monitoring/Trends}
No aligned task from \cite{rivadeneira07}. Monitoring of system status. Trends in data over time.

\subsection{Query}
Aligned to search task from \cite{rivadeneira07}.

\subsection{Navigation}
Aligned to browsing task from \cite{rivadeneira07}.

\section{Measurement of Problems}
Verbal transcripts produced and analysed.  Identified usability problems and comments are collated and compared.

\section{Design and Data analysis}

\section{Results}

\section{Conclusions}


% ------------------------------------------------------------------------


%%% Local Variables: 
%%% mode: latex
%%% TeX-master: "../thesis"
%%% End: 
