\chapter{Addressing Challenges of Tag Clouds in Software Engineering}
\ifpdf
    \graphicspath{{Chapter4/Chapter4Figs/PNG/}{Chapter4/Chapter4Figs/PDF/}{Chapter4/Chapter4Figs/}}
\else
    \graphicspath{{Chapter4/Chapter4Figs/EPS/}{Chapter4/Chapter4Figs/}}
\fi

\section{Tag Clouds vs Other Visualisations}

Where are tag clouds best or other visualisations?

\subsection{Parallel Co-ordinate Plots}

In a parallel coordinate plot, the data dimensions are represented as vertical axes arranged parallel to each other. Each data entry is shown by a single line that traverses through all  vertical axes.

A parallel coordinate plot easily data trends and differences between variables. 

Brushing - separating a data cluster by painting it a unique colour. Easy to see data relationships. Grand tour - animation looking at the data from different angles in order to look for unusual data configurations.

Scaling issue - limited to a small number of dimensions otherwise becomes overcrowded, with overlapping lines. Particular scalar/vector values cannot be emphasised.

\subsection{Treemap}

\subsection{Kiviat Chart}

\subsection{Scatterplot Matrix}


\section{Number of Variables}

What are the limits? How many variables may be mapped to a tag cloud. Not just word frequency. Is it possible to filter variables that aren't mapped.


\section{Scale}

Number of tags - 10 seems best in related work. View at package level.

\section{Distribution}

In software engineering, data is often skewed. For example, viewing LOC metric results may result in most classes sitting to one end of the graph. 

\section{Outliers}

Outliers are important in the software engineering domain.

\section{Long Identifiers}

Fat fonts. UML classes as tags.

\section{Perspective}

Providing extra detail. Zooming. Changing of perspectives.

\section{Interaction}

Exploration vs overview.  Diving into packages, classes, methods.

\section{Advantages of Tag Clouds}

Natural text representation. The label is an intrinsic part of the glyph.

% ------------------------------------------------------------------------


%%% Local Variables: 
%%% mode: latex
%%% TeX-master: "../thesis"
%%% End: 
