\chapter{Appendix C}

Details on the process of previous tag cloud evaluations using an eye-tracking device.

\section{\cite{lohmann09}}

\subsection{Results}
\begin{itemize}
	\item For all layouts, tags in the middle of a cloud attract more attention than tags at border. Circular layout increases this effect and clustered layout decreases it.
	\item Upper-left tag cloud quadrant receives the most gaze fixations.
	\item Larger tags attract more attention although this is influenced by other properties such as number of characters, tag position and neighbouring tags.
\end{itemize}

\subsection{User Tasks}
\begin{itemize}
	\item Finding a specific tag.
	\item Finding the most popular tags
	\item Finding tags that belong to a certain topic
\end{itemize}

\subsection{Tag Cloud Layout}
Two requirements of the tag clouds are that they are (1) displayed in a rectangular area and (2) that area is filled with tags as completely as possible. 

\begin{enumerate}
	\item Sequential layout, with either a horizontal or vertical arrangement of tags, sorted
alphabetically or by some other criteria (e.g., popularity, chronology, etc.)
	\item Circular layout, with the most popular tags in the center and tags with decreasing
popularities towards the borders (or vice versa)
	\item Clustered layout, in which the distance between tags follows a certain clustering
criteria (e.g., semantic relatedness) and related tags are positioned in close
proximity 
	\item No variation in the tags’ font sizes as a reference layout.
\end{enumerate}

\subsection{Tag Cloud Generation}
\begin{itemize}
	\item Equally sized rectangle with an aspect ratio of 3:2.
	\item Four tag corpora containing 100 tags each consisting of neutral terms from common knowledge areas. Aim of minimising bias from terms such as `terrorist'. 
	\item Each cloud filled with terms from a corpus. Tags mapped to 6 discrete font sizes - 27 15pt tags to one 30pt tag.
	\item Every quadrant of the tag cloud gets the same number of tags of each font size to avoid biases caused by an unbalanced presentation.
	\item In the circular layout, the 30 pt tag was placed in the middle of the cloud.
	\item In the sequential and clustered layout, the 30 pt tag was placed in another quadrant for each of the four corpora.
	\item Variation of the size and quadrant position of the tags the participants were asked for in the first task.
	\item Distribution of thematic clusters within each quadrant in the third task.
	\item Other visual features, such as font styles, weights, colours, or intensities, were kept constant in order to avoid interdependencies as reported in \cite{bateman08}
\end{itemize}

\subsection{Procedure}

\begin{enumerate}
	\item Oral explanation of tagging and tag clouds followed by a paper demonstration. Presentation of sequential layout tag clouds with alphabetical ordering.
	\item Participants randomly assigned to one of three tasks (12 participants per task).
	\item Tag clouds presented on a 17 inch TFT monitor with a screen resolution of 1280 x 1024 px. Placed in the middle of a blank screen in an area of 20 x 13.3 cm. Task is completed by clicking on a tag.
	\item Questionnaire. Layouts re-presented to aid recognition.
\end{enumerate}

\section{\cite{schrammel09b}}

\subsection{Results}
\begin{itemize}
	\item Two search strategies found - chaotic (no traceable strategy) and serial scanning (characteristic zig-zag pattern).
	\item Chaotic and serial search used independently of tag cloud layout.
	\item Typical behaviour is to switch to serial when chaotic does not yield results.
	\item No clear trend after how much time users change strategies. Users switch back and forth between two strategies.
	\item Chances for a tag with the largest font size to be fixated with a users gaze are approximately 2.5 times higher than a tag with the smallest font size. This is independent of layout.
	\item Clear influence of tag position on the amount of attention it receives from a user. Attention is strongest in upper-left quadrant and decreases via upper-right and lower-left towards the lower right quadrant.
\end{itemize}

\subsection{User Tasks}

\begin{itemize}
	\item Finding a specific tag as fast and accurately as possible.
	\item Finding a tag that belong to a certain topic.
\end{itemize}

\subsection{Tag Cloud Layout}

\begin{enumerate}
	\item Alphabetical
	\item Random - tags placed by use of random number generator.
	\item Semantic - Using the getrelated function of flickr API to retrieve a list of the tags most related to each word within the tag cloud. Based on number of co-occurring related tags, a measure for the relatedness of two
tags was calculated. An alternating least-squares algorithm to perform multidimensional scaling was used to compute a two dimensional arrangement of the tags. Finally, this value was used on the y-axis to form 7 groups of 11 resp. 10 tags each. Tags within each group were sorted according to their value on the x-axis. The result provided an 11 x 7 arrangement that was used to generate the tag cloud.
\end{enumerate}

\subsection{Tag Cloud Generation}
\begin{itemize}
	\item 304 popular tags taken from thematic clusters of flickr and randomly assigned to content groups. 
	\item Each tag cloud consisted of 76 items arranged in 7 lines, with 10 or 11 tags each line.
	\item Each cloud contained 6 very big, 11 big, 22 small and 37 very small tags (randomly assigned to each group).
\end{itemize}

\subsection{Procedure}
\begin{enumerate}
	\item Subject task completion - 15 subjects completing all tasks. Subjects are seated approximately 60 cm from eye-tracker. Tag clouds presented on a 17 inch LCD screen with a resolution of 1024 X 768 px.
	\item Eye-tracking data collected using Tobii Clearview.
	\item Manual control and correction of data.
\end{enumerate}



% ------------------------------------------------------------------------

%%% Local Variables: 
%%% mode: latex
%%% TeX-master: "../thesis"
%%% End: 
