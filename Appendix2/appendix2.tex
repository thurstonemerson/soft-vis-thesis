\chapter{Appendix B}

\section{Tools for Creation of Tag Clouds}

Libraries, frameworks and other tools for handling the creation of tag clouds.

\subsection{Web Tools}

Online tools which generate tag clouds.

	\begin{itemize}
		\item Wordle 
		\item Taxedo
		\item Tagcrowd
		\item Manyeyes
	\end{itemize}


\subsection{Processing Libraries}

Processing\footnote{\url{http://processing.org/}} library extensions which generate tag clouds. 

\begin{itemize}

	\item WordCram\footnote{\url{http://wordcram.org/}}: An open source tag cloud creation library based on Wordle.

	\item Wordookie\footnote{\url{http://code.google.com/p/wordookie/}}: An open source tag cloud creation library based on Wordle.
	 
\end{itemize}


\subsection{Javascript Libraries} 

Libraries written in javascript which generate tag clouds in a browser.

\begin{itemize}

	\item Concordle\footnote{\url{http://folk.uib.no/nfvlk/concordle/}}: Open source tag cloud creation written in javascript.

	\item TagCanvas\footnote{\url{http://www.goat1000.com/tagcanvas.php}}: Open source rotating tag cloud creation written in javascript.
	
\end{itemize}


\subsection{Java Libraries}  

Libraries written in Java to produce clouds in SWT or swing. 

\begin{itemize}
	
	\item Zest Eclipse Visualisation Toolkit\footnote{\url{http://www.eclipse.org/gef/zest/}}\footnote{ \url{http://fsteeg.com/2011/09/07/cloudio-swt-based-tag-cloud-visualization-for-zest/}}: Contains Cloudio for SWT tag cloud creation.

	\item Open Cloud\footnote{\url{http://opencloud.mcavallo.org/}}: Open source Java tag cloud creation library. 

\end{itemize}

\subsection{Google Web Toolkit Visualisation}

Create a tag cloud visualisation with Google Web Toolkit using a custom-built or open-source component. 

\section{Visualisation Libraries}

A selection a libraries written for web-based visualisation. 

\begin{itemize}
		\item InfoVis
		\item Protovis
		\item D3.js
		\item Processing.js
		\item Raphael
\end{itemize}

% ------------------------------------------------------------------------

%%% Local Variables: 
%%% mode: latex
%%% TeX-master: "../thesis"
%%% End: 
